\documentclass[a4paper,oneside,10pt]{book}
\usepackage[utf8]{inputenc}
\usepackage[T1]{fontenc}
\usepackage{longtable}
\usepackage[urlcolor=blue]{hyperref}
\usepackage{array}
\usepackage{graphicx}
\setlength{\parindent}{0pt}  
\setlength{\parskip}{4pt plus 1pt minus 0pt}  % Abstand zwischen Absaetzen
 \nonfrenchspacing
 \sloppy

\newcommand{\specialcell}[2][c]{%
  \begin{tabular}[#1]{@{}c@{}}#2\end{tabular}} 
 
 
\title{\Huge eTrice}
\author{Henkrik Renz-Reichert \and Thomas Schütz}
\begin{document}
\maketitle

\tableofcontents

\chapter{eTrice Overview}

\section{What is eTrice?}

eTrice provides an implementation of the ROOM modeling language (Real Time Object Oriented Modeling) 
together with editors, code generators for Java, C++ and C code and exemplary target middleware.

The model is defined in textual form (Xtext) with graphical editors (Graphiti) for the structural and 
behavioral (i.e. state machine) parts.

\section{Reduction of Complexity}

eTrice is all about the reduction of complexity:

\begin{itemize}
 \item structural complexity
	\begin{itemize}
\item by explicit modeling of hierarchical Actor containment, layering and inheritance \end{itemize}
\item behavioral complexity
\begin{itemize}
\item by hierachical statemachines with inheritance \end{itemize}
\item teamwork complexity
	\begin{itemize}
\item because loosely coupled Actors provide a natural way to structure team work
\item since textual model notation allows simple branching and merging \end{itemize}
\item complexity of concurrent \& distributed systems
	\begin{itemize}
\item because loosely coupled Actors are deployable to threads, processes, nodes \end{itemize}
\item complexity of variant handling and reuse (e.g. for product lines)
	\begin{itemize}
\item by composition of existing Actors to new structures
\item since Protocols and Ports make Actors replaceable
\item by inheritance for structure, behavior and Protocols
\item by making use of model level libraries  \end{itemize}
\item complexity of debugging
	 \begin{itemize}
\item model level debugging: state machine animation, data inspection and manipulation, message injection, 
generated message sequence charts
\item model checking easier for model than for code (detect errors before they occur)  \end{itemize}
\end{itemize}

\include{010-room-introduction}
\section{Working with the tutorials}

The \eTrice{} tutorials will help you to learn and understand the \eTrice{} tool and concepts. \eTrice{} supports 
several target languages. 
The first two tutorials are target language specific. The other tutorials work for all target languages. Target language specific aspects are explained for all languages. 
Currently eTrice supports Java and C. The C++ generator and runtime are currently prototypes with no tutorials. You should decide for which target language you want to run the tutorials. 

Here an overview over the tutorials:

\includegraphics[width=0.8\textwidth]{images/012-tutorial-structure.png}
% !images/012-tutorial-structure.png!

\eTrice{} generates code out of ROOM models. The generated code relies on the services of a runtime 
framework (Runtime):
\begin{itemize}
\item execution
\item communication (e.g. messaging)
\item logging
\item operating system abstraction (osal)
\end{itemize}

Additional functionality is provided as model library (Modellib): 
\begin{itemize}
\item socket server and client
\item timing service
\item standard types
\end{itemize}

All tutorial models are provided as examples.
 
The Runtime, Modellib and Tutorial projects are target language specific and will be set up in the first tutorial "Setting up the workspace for ...". 
 
\chapter{Setting up the Workspace for Java Projects}

ETrice generates code out of ROOM models. The code generator and the generated code relies on a runtime framework and on some ready to use model parts. This parts provide services like:

\begin{itemize}
\item messaging
\item logging
\item timing
\end{itemize}

Additionally some tutorial models will be provided to make it easy to start with eTrice. All this parts must be available in our workspace before you can start working. After installation of eclipse (juno) and the eTrice plug in, your workspace should look like this:  

\includegraphics[width=0.8\textwidth]{images/013-SetupWorkspace01.png}
% !images/013-SetupWorkspace01.png!

Just the \textit{eTrice} menu item is visible from the eTrice tool.
From the \textit{File} menu select \textbf{File->New->Project}

\includegraphics[width=0.8\textwidth]{images/013-SetupWorkspace02.png}
% !images/013-SetupWorkspace02.png!

Open the \textit{eTrice} tab and select \textit{eTrice Java Runtime}

Press \textit{Next} and \textit{Finish} to install the Runtime into your workspace.

\includegraphics[width=0.8\textwidth]{images/013-SetupWorkspace03.png}
% !images/013-SetupWorkspace03.png!

Do the same steps for \textit{eTrice Java Modellib} and \textit{eTrice Java Tutorials}. To avoid temporary error markers you should keep the proposed order of installation. The resulting workspace should look like this:

\includegraphics[width=0.8\textwidth]{images/013-SetupWorkspace04.png}
% !images/013-SetupWorkspace04.png!

Now workspace is set up and you can perform the tutorials or start with your work.

The tutorial models are available in the \textit{org.eclipse.etrice.tutorials} project. All tutorials are ready to generate and run without any changes. To start the code generator simply run \textbf{gen\_org.eclipse.etrice.tutorials.launch} as \textbf{gen\_org.eclipse.etrice.tutorials.launch}: 

\includegraphics[width=0.8\textwidth]{images/013-SetupWorkspace05.png}
% !images/013-SetupWorkspace05.png!

After generation for each tutorial a java file called \textbf{SubSystem\_ModelnameRunner.java} is generated. To run the model simply run this file as a java application:

\includegraphics[width=0.8\textwidth]{images/013-SetupWorkspace06.png}
% !images/013-SetupWorkspace06.png!

To stop the application type \textit{quit} in the console window.
 
\includegraphics[width=0.8\textwidth]{images/013-SetupWorkspace07.png} 
% !images/013-SetupWorkspace07.png!

Performing the tutorials will setup an dedicated project for each tutorial. Therefore there are some slight changes especially whenever a path must be set (e.g. to the model library) within your own projects. All this is described in the tutorials.

\section{Tutorial HelloWorld for Java}

\subsection{Scope}

In this tutorial you will build your first very simple \eTrice{} model. The goal is to learn the work flow of 
\eTrice{} and to understand a few basic features of ROOM. You will perform the following steps:

\begin{enumerate}
\item create a new model from scratch
\item add a very simple state machine to an actor
\item generate the source code
\item run the model
\item open the message sequence chart
\end{enumerate}

Make sure that you have set up the workspace as described in \textit{Setting up the workspace}.

\subsection{Create a new model from scratch}

The easiest way to create a new \eTrice{} Project is to use the eclipse project wizard. From the eclipse file 
menu select \textbf{File->New->Project} and create a new \eTrice{} project and name it \textbf{HelloWorld}.

\includegraphics[width=0.8\textwidth]{images/015-HelloWorld10.png}
% !images/015-HelloWorld10.png!

The wizard creates everything that is needed to create, build and run an \eTrice{} model. The resulting 
project should look like this:

\includegraphics{images/015-HelloWorld11.png}
% !images/015-HelloWorld11.png!

Within the model directory the model file \textit{HelloWorld.room} was created. Open the 
\textit{HelloWorld.room} file and delete the contents of the file. Open the content assist with Ctrl+Space 
and select \textit{model skeleton}.

\includegraphics[width=0.8\textwidth]{images/015-HelloWorld12.png}
% !images/015-HelloWorld12.png!   

Edit the template variables by typing the new names and jumping with Tab from name to name.

The resulting model code should look like this:

\begin{verbatim}
RoomModel HelloWorld {

    LogicalSystem System_HelloWorld {
        SubSystemRef subsystem : SubSystem_HelloWorld
    }

    SubSystemClass SubSystem_HelloWorld {
        ActorRef application : HelloWorldTop
    }

    ActorClass HelloWorldTop {
    }
} 
\end{verbatim}

The goal of \eTrice{} is to describe distributed systems on a logical level. In the current version not all 
elements will be used. But as prerequisite for further versions the following elements can be defined:
\begin{itemize}
\item the \textit{LogicalSystem} (currently optional)
\item at least one \textit{SubSystemClass} (mandatory)
\item at least one \textit{ActorClass} (mandatory)
\end{itemize}

The \textit{LogicalSystem} represents the complete distributed system and contains at least one 
\textit{SubSystemRef}. The \textit{SubSystemClass} represents an address space and contains at least one 
\textit{ActorRef}. The \textit{ActorClass} is the building block of which an application will be built of. 
It is in general a good idea to define a top level actor that can be used as reference within the subsystem.

The outline view of the textual ROOM editor shows the main modeling elements in an easy to navigate tree.

\includegraphics[width=0.8\textwidth]{images/015-HelloWorld02.png}
% !images/015-HelloWorld02.png!


\subsection{Create a state machine}

We will implement the Hello World code on the initial transition of the \textit{HelloWorldTop} actor. 
Therefore open the state machine editor by right clicking the \textit{HelloWorldTop} actor in the outline 
view and select \textit{Edit Behavior}.

\includegraphics{images/015-HelloWorld03.png}
% !images/015-HelloWorld03.png!

The state machine editor will be opened. Drag and drop an \textit{Initial Point} from the tool box to the 
diagram into the top level state. Drag and drop a \textit{State} from the tool box to the diagram. Confirm 
the dialogue with \textit{ok}. Select the \textit{Transition} in the tool box and draw the transition from 
the \textit{Initial Point} to the State. Open the transition dialogue by double clicking the transition 
arrow and fill in the action code.

\begin{verbatim}
	System.out.println("Hello World !");
\end{verbatim}
 
The result should look like this:

\includegraphics[width=0.8\textwidth]{images/015-HelloWorld04.png}
% !images/015-HelloWorld04.png!

Save the diagram and inspect the model file. Note that the textual representation was created after saving 
the diagram.

\includegraphics[width=0.8\textwidth]{images/015-HelloWorld05.png}
% !images/015-HelloWorld05.png!


\subsection{Build and run the model}

Now the model is finished and source code can be generated. The project wizard has created a launch 
configuration that is responsible for generating the source code. From \textit{HelloWorld/} right click 
\textbf{gen\_HelloWorld.launch} and run it as gen\_HelloWorld. All model files in the model directory will 
be generated.

\includegraphics[width=0.8\textwidth]{images/015-HelloWorld06.png}
% !images/015-HelloWorld06.png!

The code will be generated to the src-gen directory. The main function will be contained in 
\textbf{SubSystem\_HelloWorldRunner.java}. Select this file and run it as Java application.

\includegraphics{images/015-HelloWorld07.png}
% !images/015-HelloWorld07.png!


The Hello World application starts and the string will be printed on the console window. To stop the 
application the user must type \textbf{quit} in the console window.

\includegraphics[width=0.8\textwidth]{images/015-HelloWorld08.png}
% !images/015-HelloWorld08.png!

\subsection{Open the Message Sequence Chart}

During runtime the application produced a MSC and wrote it to a file. Open 
HelloWorld/tmp/log/SubSystem\_HelloWorld\_Async.seq using Trace2UML (it is open source and can be obtained 
from  http://trace2uml.tigris.org/). You should see something like this:

\includegraphics[width=0.8\textwidth]{images/015-HelloWorld09.png}
% !images/015-HelloWorld09.png!


\subsection{Summary}

Now you have generated your first \eTrice{} model from scratch. You can switch between diagram editor and 
model (.room file) and you can see what will be generated during editing and saving the diagram files. 
You should take a look at the generated source files to understand how the state machine is generated and 
the life cycle of the application. The next tutorials will deal with more complex hierarchies in structure 
and behavior.

\chapter{Tutorial Blinky (Java)}

\section{Scope}

This tutorial describes how to use the \textit{TimingService}, how to combine a generated model with 
manual code and how to model a hierarchical state machine. The idea of the tutorial is to switch a LED on 
and off. The behavior of the LED should be: blinking in a one second interval for 5 seconds, stop blinking 
for 5 seconds, blinking, stop,...  
For this exercise we will use a little GUI class that will be used in more sophisticated tutorials too. 
The GUI simulates a pedestrian traffic crossing. For now, just a simple LED simulation will be used from 
the GUI. 

After the exercise is created you must copy the GUI to your src directory (see below).

The package contains four java classes which implements a small window with a 3-light traffic light which 
simulates the signals for the car traffic and a 2-light traffic light which simulates the pedestrian 
signals.

The GUI looks like this:

\includegraphics{images/020-Blinky08.png}
% !images/020-Blinky08.png!

Within this tutorial we will just toggle the yellow light.

You will perform the following steps:

\begin{enumerate}
\item create a new model from scratch
\item define a protocol
\item create an actor structure
\item create a hierarchical state machine
\item use the predefined \textit{TimingService}
\item combine manual code with generated code
\item build and run the model
\item open the message sequence chart
\end{enumerate}

\section{Create a new model from scratch}

Remember the exercise \textit{HelloWorld}.
Create a new eTrice project and name it \textit{Blinky}.

To use the GUI please copy the package \textit{org.eclipse.etrice.tutorials.PedLightGUI} from 
\textit{org.eclipse.etrice.tutorials/src} to your <em>src</em> directory \textit{Blinky/src}. For this tutorial 
you must remove the error markers by editing the file \textit{PedestrianLightWndNoTcp.java}. Appropriate 
comments are provided to remove the error markers for this turorial.

Open the \textit{Blinky.room} file and copy the following code into the file or use content assist to 
create the model.

\begin{verbatim} 
RoomModel Blinky {

    LogicalSystem System_Blinky {
        SubSystemRef subsystem : SubSystem_Blinky
    }

    SubSystemClass SubSystem_Blinky {
        ActorRef application : BlinkyTop
    }

    ActorClass BlinkyTop {
    }
}
\end{verbatim}

\section{Add two additional actor classes}

Position the cursor outside any class definition and right click the mouse within the editor window. From 
the context menu select \textit{Content Assist}  

\includegraphics[width=0.8\textwidth]{images/020-Blinky02.png}
% !images/020-Blinky02.png!

Select \textit{ActorClass - actor class skeleton} and name it \textit{Blinky}.

\includegraphics[width=0.8\textwidth]{images/020-Blinky01.png}
% !images/020-Blinky01.png! 

Repeat the described procedure and name the new actor \textit{BlinkyController}.

With Ctrl+Shift+F you can beautify the model code. 

Save the model and visit the outline view.

\section{Create a new protocol}

With the help of \textit{Content Assist} create a \textit{ProtocolClass} and name it 
\textit{BlinkyControlProtocol}.
Inside the brackets use the \textit{Content Assist} (CTRL+Space) to create two incoming messages called 
\textit{start} and \textit{stop}.

The resulting code should look like this:

\includegraphics[width=0.8\textwidth]{images/020-Blinky03.png}
% !images/020-Blinky03.png!

With Ctrl-Shift+F or selecting \textit{Format} from the context menu you can format the text. Note that 
all elements are displayed in the outline view.

\section{Import the Timing Service}

Switching on and off the LED is timing controlled. The timing service is provided from the model library 
and must be imported before it can be used from the model.

This is the first time you use an element from the modellib. Make sure that your Java Build Path has the 
appropriate entry to the modellib. Otherwise the jave code, which will be generated from the modellib, can 
not be referenced.
(right click to \textit{Blinky} and select properties. Select the \textit{Java Build Path} tab) 
  
\includegraphics[width=0.8\textwidth]{images/020-Blinky16.png}
% !images/020-Blinky16.png! 

After the build path is set up return to the model and navigate the cursor at the beginning of the model 
and import the timing service: 

\begin{small}
\begin{verbatim}
RoomModel Blinky {
    
    import room.basic.service.timing.* from 
		"../../org.eclipse.etrice.modellib/models/TimingService.room" 
    
    LogicalSystem System_Blinky {
        SubSystemRef subsystem: SubSystem\_Blinky
    }
}
...     
\end{verbatim}
\end{small}

Make sure that the path fits to your folder structure. The original tutorial code is different due to the 
folder structure.  

Now it can be used within the model. Right click to \textbf{SubSystem\_Blinky} within the outline view. 
Select \textit{Edit Structure}. The \textit{application} is already referenced in the subsystem. Drag and 
Drop an \textit{ActorRef} to the \textbf{SubSystem\_Blinky} and name it \textit{timingService}. From the 
actor class drop down list select \textit{room.basic.service.timing.ATimingService}. Draw a 
\textit{LayerConnection} from \textit{application} to each service provision point (SPP) of the 
\textit{timingService}. The resulting structure should look like this:

\includegraphics[width=0.8\textwidth]{images/020-Blinky06.png}
% !images/020-Blinky06.png! 

The current version of eTrice does not provide a graphical element for a service access point (SAP). 
Therefore the SAPs to access the timing service must be added in the .room file. Open the 
\textit{Blinky.room} file and navigate to the \textit{Blinky} actor. Add the following line to the 
structure of the actor:

\begin{verbatim}SAP timer: room.basic.service.timing.PTimeout \end{verbatim}

Do the same thing for \textit{BlinkyController}.

The resulting code should look like this:

\includegraphics[width=0.8\textwidth]{images/020-Blinky07.png}
% !images/020-Blinky07.png!


\section{Finish the model structure}

From the outline view right click to \textit{Blinky} and select \textit{Edit Structure}. Drag and Drop an 
\textit{Interface Port} to the boarder of the \textit{Blinky} actor. Note that an interface port is not 
possible inside the actor. Name the port \textit{ControlPort} and select \textit{BlinkyControlProtocol} 
from the drop down list. Uncheck \textit{Conjugated} and \textit{Is Relay Port}. Click \textit{ok}. The 
resulting structure should look like this:

\includegraphics[width=0.8\textwidth]{images/020-Blinky04.png}
% !images/020-Blinky04.png!

Repeat the above steps for the \textit{BlinkyController}. Make the port \textit{Conjugated}

Keep in mind that the protocol defines \textit{start} and \textit{stop} as incoming messages. 
\textit{Blinky} receives this messages and therefore \textit{Blinky}'s \textit{ControlPort} must be a 
regular port and \textit{BlinkyController}'s \textit{ControlPort} must be a conjugated port.


From the outline view right click \textit{BlinkyTop} and select \textit{Edit Structure}.

Drag and Drop an \textit{ActorRef} inside the \textit{BlinkyTop} actor. Name it \textit{blinky}. From the 
actor class drop down list select \textit{Blinky}. Do the same for \textit{controller}. Connect the ports 
via the binding tool. The resulting structure should look like this:

\includegraphics[width=0.8\textwidth]{images/020-Blinky05.png}
% !images/020-Blinky05.png!

\section{Implement the Behavior}

The application should switch on and off the LED for 5 seconds in a 1 second interval, then stop blinking 
for 5 seconds and start again. To implement this behavior we will implement two FSMs. One for the 1 second 
interval and one for the 5 second interval. The 1 second blinking should be implemented in 
\textit{Blinky}. The 5 second interval should be implemented in \textit{BlinkyController}. First implement 
the Controller.

Right click to \textit{BlinkyController} and select \textit{Edit Behavior}.
Drag and Drop the \textit{Initial Point} and two \textit{States} into the top state. Name the states 
\textit{on} and \textit{off}. 
Use the \textit{Transition} tool to draw transitions from \textit{init} to \textit{on} from \textit{on} to 
\textit{off} and from \textit{off} to \textit{on}.

Open the transition dialog by double click the arrow to specify the trigger event and the action code of 
each transition. Note that the initial transition does not have a trigger event.

The transition dialog should look like this:

\includegraphics[width=0.8\textwidth]{images/020-Blinky09.png}
% !{width=500px}images/020-Blinky09.png! 

The defined ports will be generated as a member attribute of the actor class from type of the attached 
protocol. So, to send e message you must state \textit{port.message(param);}. In this example 
\textit{ControlPort.start()} sends the \textit{start} message via the \textit{ControlPort} to the outside 
world. Assuming that \textit{Blinky} is connected to this port, the message will start the one second 
blinking FSM. It is the same thing with the \textit{timer}. The SAP is also a port and follows the same 
rules. So it is clear that \textit{timer.Start(5000);} will send the \textit{Start} message to the timing 
service. The timing service will send a \textit{timeoutTick} message back after 5000ms.

Within each transition the timer will be restarted and the appropriate message will be sent via the 
\textit{ControlPort}. 

The resulting state machine should look like this:
(Note that the arrows peak changes if the transition contains action code.)

\includegraphics[width=0.8\textwidth]{images/020-Blinky10.png}
% !images/020-Blinky10.png!

Save the diagram and inspect the \textit{Blinky.room} file. The \textit{BlinkyController} should look like 
this:

\includegraphics[width=0.8\textwidth]{images/020-Blinky11.png}
% !images/020-Blinky11.png! 
 
Now we will implement \textit{Blinky}. Due to the fact that \textit{Blinky} interacts with the GUI class a 
view things must to be done in the model file.

Double click \textit{Blinky} in the outline view to navigate to \textit{Blinky} within the model file.
Add the following code:
(type it or simply copy it from the tutorial project)

\includegraphics[width=0.8\textwidth]{images/020-Blinky12.png}
% !images/020-Blinky12.png! 

\textit{usercode1} will be generated at the beginning of the file, outside the class definition. 
\textit{usercode2} will be generated within the class definition. The code imports the GUI class and 
instantiates the window class. Attributes for the carLights and pedLights will be declared to easily 
access the lights in the state machine.
The Operation \textit{destroyUser()} is a predefined operation that will be called during shutdown of the 
application. Within this operation, cleanup of manual coded classes can be done.
 
Now design the FSM of \textit{Blinky}. Remember, as the name suggested \textit{blinking} is a state in 
which the LED must be switched on and off. We will realize that by an hierarchical FSM in which the 
\textit{blinking} state has two sub states.

Open the behavior diagram of \textit{Blinky} by right clicking the \textit{Blinky} actor in the outline 
view. Create two states named \textit{blinking} and \textit{off}. Right click to \textit{blinking} and 
create a subgraph.

\includegraphics[width=0.8\textwidth]{images/020-Blinky13.png}
% !images/020-Blinky13.png!

Create the following state machine. The trigger events between \textit{on} and \textit{off} are the 
\textit{timeoutTick} from the \textit{timer} port. 

\includegraphics[width=0.8\textwidth]{images/020-Blinky14.png}
% !images/020-Blinky14.png!

Create entry code for both states by right clicking the state and select \textit{Edit State...}

Entry code of \textit{on} is:

\begin{verbatim}
timer.Start(1000);
carLights.setState(TrafficLight3.YELLOW); 
\end{verbatim}

 
Entry code  of \textit{off} is:

\begin{verbatim}
timer.Start(1000);
carLights.setState(TrafficLight3.OFF);
\end{verbatim}

Navigate to the Top level state by double clicking the \textit{/blinking} state. Create the following 
state machine:

\includegraphics[width=0.8\textwidth]{images/020-Blinky15.png}
% !images/020-Blinky15.png!

The trigger event from \textit{off} to \textit{blinking} is the \textit{start} event from the 
\textit{ControlPort}.The trigger event from \textit{blinking} to \textit{off} is the \textit{stop} event 
from the \textit{ControlPort}.
Note: The transition from \textit{blinking} to \textit{off} is a so called group transition. This is a 
outgoing transition from a super state (state with sub states) without specifying the concrete leave state 
(state without sub states). An incoming transition to a super state is called history transition.   

Action code of the init transition is:

\begin{verbatim}
carLights = light.getCarLights();
pedLights = light.getPedLights();
carLights.setState(TrafficLight3.OFF);
pedLights.setState(TrafficLight2.OFF);
\end{verbatim}

Action code from \textit{blinking} to \textit{off} is:

\begin{verbatim}
timer.Kill();
carLights.setState(TrafficLight3.OFF); 
\end{verbatim}

The model is complete now. You can run and debug the model as described in getting started. Have fun.

The complete model can be found in /org.eclipse.etrice.tutorials/model/Blinky.

\section{Summary}

Run the model and take a look at the generated MSCs. Inspect the generated code to understand the runtime 
model of eTrice. Within this tutorial you have learned how to create a hierarchical FSM with group 
transitions and history transitions and you have used entry code. You are now familiar with the basic 
features of eTrice. The further tutorials will take this knowledge as a precondition.

\chapter{Tutorial Sending Data (Java)}

\section{Scope}

This tutorial shows how data will be sent in a \eTrice{} model. Within the example you will create two actors 
(MrPing and MrPong). MrPong will simply loop back every data it received.
MrPing will send data and verify the result.   

You will perform the following steps:

\begin{enumerate}
\item create a new model from scratch
\item create a data class
\item define a protocol with attached data
\item create an actor structure
\item create two simple state machines
\item build and run the model
\end{enumerate}

\section{Create a new model from scratch}

Remember exercise \textit{HelloWorld}.
Create a new \eTrice{} project and name it \textit{SendingData}.
Open the \textit{SendingData.room} file and copy the following code into the file or use content assist to 
create the model.


\begin{verbatim} 
RoomModel SendingData {
    LogicalSystem SendingData_LogSystem {
        SubSystemRef SendingDataAppl:SendingData_SubSystem 
    }
    SubSystemClass SendingData_SubSystem {
        ActorRef SendigDataTopRef:SendingDataTop 
    }
    ActorClass SendingDataTop {
    }
}
\end{verbatim}

\section{Add a data class}

Position the cursor outside any class definition and right click the mouse within the editor window. From 
the context menu select \textit{Content Assist} (or Ctrl+Space).  

\includegraphics[width=0.8\textwidth]{images/025-SendingData01.png}
% !images/025-SendingData01.png!

Select \textit{DataClass - data class skeleton} and name it \textit{DemoData}.
Remove the operations and add the following Attributes:

\begin{verbatim}
DataClass DemoData {
    Attribute int32Val: int32 = "4711"
    Attribute int8Array [ 10 ]: int8 = "{1,2,3,4,5,6,7,8,9,10}"
    Attribute float64Val: float64 = "0.0"
    Attribute stringVal: string = "\"empty\""
}
\end{verbatim}

Save the model and visit the outline view.
Note that the outline view contains all data elements as defined in the model. 

\section{Create a new protocol}

With the help of \textit{Content Assist} create a \textit{ProtocolClass} and name it 
\textit{PingPongProtocol}. Create the following messages:

\begin{verbatim} 
ProtocolClass PingPongProtocol {
    incoming {
        Message ping(data: DemoData)
        Message pingSimple(data:int32)
    }
    outgoing {
        Message pong(data: DemoData)
        Message pongSimple(data:int32)
    }
}    
\end{verbatim}

\section{Create MrPing and MrPong Actors}

With the help of \textit{Content Assist} create two new actor classes and name them \textit{MrPing} and 
\textit{MrPong}. The resulting model should look like this:

\begin{verbatim}
RoomModel SendingData {

    LogicalSystem SendingData_LogSystem {
        SubSystemRef SendingDataAppl: SendingData_SubSystem
    }

    SubSystemClass SendingData_SubSystem {
        ActorRef SendigDataTopRef: SendingDataTop
    }

    ActorClass SendingDataTop { }

    DataClass DemoData {
        Attribute int32Val: int32 = "4711"
        Attribute int8Array [ 10 ]: int8 = "{1,2,3,4,5,6,7,8,9,10}"
        Attribute float64Val: float64 = "0.0"
        Attribute stringVal: string = "\"empty\""
    }

    ProtocolClass PingPongProtocol {
        incoming {
            Message ping(data: DemoData)
            Message pingSimple(data: int32)
        }
        outgoing {
            Message pong(data: DemoData)
            Message pongSimple(data: int32)
        }
    }

    ActorClass MrPing {
        Interface { }
        Structure { }
        Behavior { }
    }

    ActorClass MrPong {
        Interface { }
        Structure { }
        Behavior { }
    }
} 

\end{verbatim}

The outline view should look like this:

\includegraphics{images/025-SendingData03.png}
% !images/025-SendingData03.png!

\section{Define Actor Structure and Behavior}

Save the model and visit the outline view. Within the outline view, right click on the \textit{MrPong} 
actor and select \textit{Edit Structure}. Select an \textit{Interface Port} from the toolbox and add it to 
MrPong. Name the Port \textit{PingPongPort} and select the \textit{PingPongProtocol}.

\includegraphics[width=0.8\textwidth]{images/025-SendingData02.png}
% !images/025-SendingData02.png!

Do the same with MrPing but mark the port as \textit{conjugated}

\subsection{Define MrPongs behavior}

Within the outline view, right click MrPong and select \textit{Edit Behavior}. Create the following state 
machine:

\includegraphics[width=0.8\textwidth]{images/025-SendingData04.png}
% !images/025-SendingData04.png!

The transition dialogues should look like this:
For \textit{ping}:

\includegraphics[width=0.8\textwidth]{images/025-SendingData05.png}
% !images/025-SendingData05.png!

For \textit{pingSimple}:

\includegraphics[width=0.8\textwidth]{images/025-SendingData06.png}
% !images/025-SendingData06.png!


\subsection{Define MrPing behavior}

Within the outline view double click MrPing. Navigate the cursor to the behavior of MrPing. With the help 
of content assist create a new operation.

\includegraphics[width=0.8\textwidth]{images/025-SendingData07.png}
% !images/025-SendingData07.png!

Name the operation \textit{printData} and define the DemoData as a parameter.

Fill in the following code:

\begin{small}
\begin{verbatim}
Operation printData(d: DemoData) : void {
            "System.out.printf(\"d.int32Val: %d\\n\",d.int32Val);"
            "System.out.printf(\"d.float64Val: %f\\n\",d.float64Val);"
            "System.out.printf(\"d.int8Array: \");"
            "for(int i = 0; i<d.int8Array.length; i++) {"
            "System.out.printf(\"%d \",d.int8Array[i]);}"
            "System.out.printf(\"\\nd.stringVal: %s\\n\",d.stringVal);"
}
\end{verbatim}
\end{small}

For MrPing create the following state machine:
(Remember that you can copy and paste the action code from the tutorial directory.)

\includegraphics[width=0.8\textwidth]{images/025-SendingData08.png}
% !images/025-SendingData08.png!

The transition dialogues should look like this:

For \textit{init}:

\includegraphics[width=0.8\textwidth]{images/025-SendingData09.png}
% !images/025-SendingData09.png!

For \textit{wait1}:

\includegraphics[width=0.8\textwidth]{images/025-SendingData10.png}
% !images/025-SendingData10.png!

For \textit{next}:

\includegraphics[width=0.8\textwidth]{images/025-SendingData11.png}
% !images/025-SendingData11.png!

For \textit{wait2}:

\includegraphics[width=0.8\textwidth]{images/025-SendingData12.png}
% !images/025-SendingData12.png!

\section{Define the top level}

Open the Structure from SendingDataTop and add MrPing and MrPong as a reference. Connect the ports.

\includegraphics[width=0.8\textwidth]{images/025-SendingData13.png}
% !images/025-SendingData13.png!

\begin{flushleft}The model is finished now and can be found in 
/org.eclipse.etrice.tutorials/model/SendingData.\end{flushleft}

\section{Generate and run the model}

Generate the code by right click to \textbf{gen\_SendingData.launch} and run it as 
\textbf{gen\_SendingData}. Run the model. 
The output should look like this:

\begin{verbatim}
type 'quit' to exit
/SendingData_SubSystem/SendigDataTopRef/ref0 -> waitForPongSimple
/SendingData_SubSystem/SendigDataTopRef/ref1 -> looping
/SendingData_SubSystem/SendigDataTopRef/ref1 -> looping
data: 1
/SendingData_SubSystem/SendigDataTopRef/ref0 -> waitForPongSimple
/SendingData_SubSystem/SendigDataTopRef/ref1 -> looping
data: 2
/SendingData_SubSystem/SendigDataTopRef/ref0 -> waitForPongSimple
/SendingData_SubSystem/SendigDataTopRef/ref1 -> looping
data: 3
/SendingData_SubSystem/SendigDataTopRef/ref0 -> waitForPongSimple
/SendingData_SubSystem/SendigDataTopRef/ref1 -> looping
data: 4
/SendingData_SubSystem/SendigDataTopRef/ref0 -> waitForPongSimple
/SendingData_SubSystem/SendigDataTopRef/ref1 -> looping
data: 5
/SendingData_SubSystem/SendigDataTopRef/ref0 -> waitForPongSimple
/SendingData_SubSystem/SendigDataTopRef/ref1 -> looping
data: 6
/SendingData_SubSystem/SendigDataTopRef/ref0 -> waitForPongSimple
/SendingData_SubSystem/SendigDataTopRef/ref1 -> looping
data: 7
/SendingData_SubSystem/SendigDataTopRef/ref0 -> waitForPongSimple
/SendingData_SubSystem/SendigDataTopRef/ref1 -> looping
data: 8
/SendingData_SubSystem/SendigDataTopRef/ref0 -> waitForPongSimple
/SendingData_SubSystem/SendigDataTopRef/ref1 -> looping
data: 9
/SendingData_SubSystem/SendigDataTopRef/ref0 -> waitForPongSimple
/SendingData_SubSystem/SendigDataTopRef/ref1 -> looping
data: 10
/SendingData_SubSystem/SendigDataTopRef/ref0 -> waitForPong
/SendingData_SubSystem/SendigDataTopRef/ref1 -> looping
/SendingData_SubSystem/SendigDataTopRef/ref1 -> looping
d.int32Val: 4711
d.float64Val: 0,000000
d.int8Array: 1 2 3 4 5 6 7 8 9 10 
d.stringVal: empty
/SendingData_SubSystem/SendigDataTopRef/ref0 -> waitForPong
d.int32Val: 815
d.float64Val: 3,141234
d.int8Array: 100 101 102 103 104 105 106 107 108 109 
d.stringVal: some contents
/SendingData_SubSystem/SendigDataTopRef/ref0 -> waitForPong
quit
echo: quit
\end{verbatim}

\section{Summary}

Within the first loop an integer value will be incremented by \textit{MrPong} and sent back to 
\textit{MrPing}. As long as the guard is true \textit{MrPing} sends back the value.

Within the \textit{next} transition, \textit{MrPing} creates a data class and sends the default values. 
Then \textit{MrPing} changes the values and sends the class again. At this point you should note that 
during the send operation, a copy of the data class will be created and sent. Otherwise it would not be 
possible to send the same object two times, even more it would not be possible to send a stack object at 
all. This type of data passing is called \textit{sending data by value}.
However, for performance reasons some applications requires \textit{sending data by reference}. In this 
case the user is responsible for the life cycle of the object. In Java the VM takes care of the life cycle 
of an object. This is not the case for C/C++. Consider that a object which is created within a transition 
of a state machine will be destroyed when the transition is finished. The receiving FSM would receive an 
invalid reference. Therefore care must be taken when sending references.      

For sending data by reference you simply have to add the keyword \textit{ref} to the protocol definition.
 
\begin{verbatim}Message ping(data: DemoData ref)\end{verbatim}

Make the test and inspect the console output.

\chapter{Tutorial Pedestrian Lights (Java)}

\section{Scope}

The scope of this tutorial is to demonstrate how to receive model messages from outside the model. Calling 
methods which are not part of the model is simple and you have already done this within the blinky 
tutorial (this is the other way round: model => external code). Receiving events from outside the model is 
a very common problem and a very frequently asked question. Therefore this tutorial shows how an external 
event (outside the model) can be received by the model.

This tutorial is not like hello world or blinky. Being familiar with the basic tool features is mandatory 
for this tutorial. The goal is to understand the mechanism not to learn the tool features.

The idea behind the exercise is, to control a Pedestrian crossing light. We will use the same GUI as for 
the blinky tutorial but now we will use the \textit{REQUEST} button to start a FSM, which controls the 
traffic lights.

\includegraphics{images/020-Blinky08.png}
% !images/020-Blinky08.png!

The \textit{REQUEST} must lead to a model message which starts the activity of the lights.

There are several possibilities to receive external events (e.g. TCP/UDP Socket, using OS messaging 
mechanism), but the easiest way is, to make a port usable from outside the model. To do that a few steps 
are necessary:
\begin{enumerate}
\item specify the messages (within a protocol) which should be sent into the model
\item model an actor with a port (which uses the specified protocol) and connect the port to the receiver 
\item the external code should know the port (import of the port class)
\item the external code should provide a registration method, so that the actor is able to allow access to 
this port
\item the port can be used from the external code
\end{enumerate}

\section{Setup the model}

\begin{itemize}
\item Use the \textit{New Model Wizzard} to create a new eTrice project and name it 
\textit{PedLightsController}.
\item Copy the package \textit{org.eclipse.etrice.tutorials.PedLightGUI} to your \textit{src} directory 
(see blinky tutorial).
\item In PedestrianLightWndNoTcp.jav uncomment line 15 (import), 36, 122 (usage) and 132-134 
(registration). The error markers will disappear after the code is generated from the model.
\item \begin{flushleft}Copy the model from /org.eclipse.etrice.tutorials/model/PedLightsController to your 
model file, or run the model directly in the tutorial directory.\end{flushleft} 
\item Adapt the import statement to your path.
\end{itemize}

\begin{small}
\begin{verbatim} 
import room.basic.service.timing.* from 
	"../../org.eclipse.etrice.modellib/models/TimingService.room" 
\end{verbatim}
\end{small}

\begin{itemize}
\item Generate the code from the model.
\item Add the org.eclipse.etrice.modellib to the Java Class Path of your project.
\item All error markers should be disappeared and the model should be operable. 
\item Arrange the Structure and the Statemachines to understand the model
\end{itemize}

\includegraphics[width=0.8\textwidth]{images/030-PedLights01.png}
% !images/030-PedLights01.png!
The \textit{GuiAdapter} represents the interface to the external code. It registers its 
\textit{ControlPort} by the external code.

\includegraphics[width=0.8\textwidth]{images/030-PedLights02.png}
% !images/030-PedLights02.png!
Visit the initial transition to understand the registration. The actor handles the incoming messages as 
usual and controls the traffic lights as known from blinky. 

\includegraphics[width=0.8\textwidth]{images/030-PedLights03.png}
% !images/030-PedLights03.png!
The \textit{Controller} receives the \textit{start} message and controls the timing of the lights. Note 
that the \textit{start} message will be sent from the external code whenever the \textit{REQUEST} button 
is pressed.

\begin{itemize}
\item  Visit the model and take a closer look to the following elements:
\begin{enumerate}
\item PedControlProtocol => notice that the start message is defined as usual
\item Initial transition of the \textit{GuiAdapter} => see the registration
\item The \textit{Controller} => notice that the \textit{Controller} receives the external message (not 
the \textit{GuiAdapter}). The \textit{GuiAdapter} just provides its port and handles the incoming messages.
\item Visit the hand written code => see the import statement of the protocol class and the usage of the 
port.
\end{enumerate}
\item Generate and test the model
\item Take a look at the generated MSC => notice that the start message will shown as if the 
\textit{GuiAdapter} had sent it.
\end{itemize}

\includegraphics[width=0.8\textwidth]{images/030-PedLights04.png}
% !images/030-PedLights04.png!

\section{Why does it work and why is it safe?}

The tutorial shows that it is generally possible to use every port from outside the model as long as the 
port knows its peer. This is guaranteed by describing protocol and the complete structure (especially the 
bindings) within the model. 
The only remaining question is: Why is it safe and does not violate the \textbf{run to completion} 
semantic. To answer this question, take a look at the \textit{MessageService.java} from the runtime 
environment. There you will find the receive method which puts each message into the queue. 

\begin{verbatim}
    @Override
    public synchronized void receive(Message msg) {
        if (msg!=null) {
            messageQueue.push(msg);
            notifyAll(); // wake up thread to compute message
        }
    }
\end{verbatim}

This method is synchronized. That means, regardless who sends the message, the queue is secured. If we 
later on (e.g. for performance reasons in C/C++) distinguish between internal and external senders (same 
thread or not), care must be taken to use the external (secure) queue.

\include{032-setting-up-the-workspace-c}
\include{034-getting-started-c}
\chapter{Tutorial Remove C-Comment ( C )}

\section{Scope}

In this tutorial you will create a more complex model. The model implements a simple parser that removes comments (block comments and line comments) from a C source file. Therefore we will create two actors. One actor is responsible to perform the file operations, while the second actor implements the parser.

You will perform the following steps:

\begin{enumerate}
\item create a new model from scratch for C
\item define a protocol
\item define your own data type
\item create the structure and the behavior by yourself
\item generate, build and run the model
\end{enumerate}

Make sure that you have set up the workspace as described in \textit{Setting up the Workspace for C Projects}.

\section{Create a new model from scratch}

Remember the following steps from the previous tutorials:
\begin{itemize}
\item select the \textit{C/C++} perspective
\item From the main menue select \textit{File->New->C Project}
\item Name the project \textit{RemoveComment}
\item Project type is \textit{Executable / Empty C Project}
\item Toolchain is \textit{MinGW}
\item Add the folder \textit{model}
\item Add the model file and name it \textit{RemoveComment.room}
\item Add the Xtext nature.
\end{itemize}

The workspace should look like this:

\includegraphics{images/036-RemoveCommentC01.png}
% !images/036-RemoveCommentC01.png!

Create a launch configuration for the C generator and add the include path and library as described in \textit{HelloWorldC}.

The workspace should look like this:

\includegraphics{images/036-RemoveCommentC02.png}
% !images/036-RemoveCommentC02.png!

Now the model is created and all settings for the code generator, compiler and linker are done.


\section{Create your own data type}

The planed application should read a C source file and remove the comments. Therefore we need a file descriptor which is not part of the basic C types. The type for the file descriptor for MinGW is \textit{FILE}. To make this type available on the model level, you have to declare the type. 

Open the file \textit{Types.room} from \textit{org.eclipse.modellib.c} and take a look at the declaration of \textit{string} (last line) which is not a basic C type.

\textit{PrimitiveType string:ptCharacter -> charPtr default "0"}

With this declaration, you make the \textit{string} keyword available on model level as a primitive type. This type will be translated to \textit{charPtr} in your C sources. \textit{charPtr} is defined in \textit{etDatatypes.h}. This header file is platform specific (\textit{generic}). With this mechanism you can define your own type system on model level and map the model types to specific target/platform types. 

To not interfere with other models, we will declare the type direct in the model.
Add the following line to your model:

\begin{small}
\begin{verbatim}
RoomModel RemoveComment {
	import room.basic.types.* from 
		"../../../org.eclipse.etrice.modellib.c/model/Types.room"
	
	PrimitiveType file:ptInteger -> FILE default "0"
\end{verbatim}
\end{small}

\textit{FILE} is the native type for MinGW. Therefore you don't need a mapping within \textit{etDatatypes.h}. If your model should be portable across different platforms you should not take this shortcut.
 
\section{Create the model}

Due to the former tutorials you should be familiar with the steps to create the model with protocols, actors and state machines.

The basic idea of the exercise is to create a file reader actor, which is responsible to open, close and read characters from the source file. Another actor receives the characters and filters the comments (parser). The remaining characters (pure source code) should be print out. 

Remember the logical steps: 
\begin{itemize}
\item create the model by the help of content assist (CTRL Space)
\item name the model, subsystem and top level actor
\item define the protocol (in this case it should be able to send a char, and to request the next char from the file reader)
\item create the structure (file reader and parser with an appropriate port, create the references and connect the ports)
\item create the state machines
\end{itemize}

Try to create the model by yourself and take the following solution as an example.

Structure:

\includegraphics[width=0.8\textwidth]{images/036-RemoveCommentC04.png}
% !images/036-RemoveCommentC04.png!

File reader FSM:

\includegraphics[width=0.8\textwidth]{images/036-RemoveCommentC05.png}
% !images/036-RemoveCommentC05.png!

Parser FSM:

\includegraphics[width=0.8\textwidth]{images/036-RemoveCommentC06.png}
% !images/036-RemoveCommentC06.png!

The complete model can be found in \textit{org.eclipse.etrice.tutorials.c}

Take a look at the file attribute of the file reader. 

\begin{verbatim}
Attribute f:file ref
\end{verbatim}

\textit{fopen} expects a \textit{FILE *}. \textit{f:file ref} declares a variable \textit{f} from type reference to \textit{file}, which is a pointer to \textit{FILE}.


\section{Generate, build and run the model}

Before you can run the model you should copy one of the generated C source files into the project folder and name it \textit{test.txt}. 

\includegraphics{images/036-RemoveCommentC07.png}
% !images/036-RemoveCommentC07.png!

Generate, build and run the model.

Your output should start like this:

\includegraphics[width=0.8\textwidth]{images/036-RemoveCommentC08.png}
% !images/036-RemoveCommentC08.png!


\section{Summary}

This tutorial should help you to train the necessary steps to create a C model. By the way you have seen how to create your own type system for a real embedded project. An additional aspect was to show how simple it is to separate different aspects of the required functionality by the use of actors and protocols and make them reusable.

\chapter{ROOM Concepts}

This chapter gives an overview over the ROOM language elements and their textual and graphical notation.
The formal ROOM grammar based on Xtext (EBNF) you can find here: "ROOM Grammar":http://git.eclipse.org/c/etrice/org.eclipse.etrice.git/tree/plugins/org.eclipse.etrice.core.room/src/org/eclipse/etrice/core/Room.xtext

\section{Actors}

\subsection{Description}
 
The actor is the basic structural building block for building systems with ROOM. An actor can be refined hierarchically and thus can be of arbitrarily large scope. Ports define the interface of an actor. An Actor can also have a behavior usually defined by a finite state machine.

\subsection{Motivation}

\begin{itemize}
\item Actors enable the construction of hierarchical structures by composition and layering
\item Actors have their own logical thread of execution
\item Actors can be freely deployed
\item Actors define potentially reusable blocks
\end{itemize}

\subsection{Notation}


\begin{table}
\caption{Actor Class Notation}
\begin{tabular}{|l|l|l|}
\hline
 \textbf{Element} & \textbf{Graphical Notation} & \textbf{Textual Notation} \\ \hline
  ActorClass & \includegraphics{images/040-ActorClassNotation.png} & \includegraphics{images/040-ActorClassTextualNotation.png} \\ \hline
  ActorRef & \includegraphics{images/040-ActorReferenceNotation.png} & \includegraphics{images/040-ActorReferenceTextualNotation.png} \\ \hline
\end{tabular}
\end{table}

% <table title="Actor Class Notation" frame="box" border="2" cellpadding="3" cellspacing="0" >
	% <tr>
		% <td align="center">*Element*</td>
		% <td align="center">*Graphical Notation*</td>
		% <td align="center">*Textual Notation*</td>
	% </tr>
	% <tr>
		% <td>ActorClass</td>
		% <td>!images/040-ActorClassNotation.png!</td>
		% <td>!images/040-ActorClassTextualNotation.png!</td>
	% </tr>
	% <tr>
		% <td>ActorRef</td>
		% <td>!images/040-ActorReferenceNotation.png!</td>
		% <td>!images/040-ActorReferenceTextualNotation.png!</td>
	% </tr>
% </table> 


\subsection{Details}

\subsubsection{Actor Classes, Actor References, Ports and Bindings}

An \textbf{ActorClass} defines the type (or blueprint) of an actor. Hierarchies are built by ActorClasses that contain \textbf{ActorReferences} which have another ActorClass as type. The interface of an ActorClass is always defined by Ports. The ActorClass can also contain Attributes, Operations and a finite state machine. 

\textbf{External Ports} define the external interface of an actor and are defined in the *Interface* section of the ActorClass.

\textbf{Internal Ports} define the internal interface of an actor and are defined in the *Structure* section of the ActorClass.

\textbf{Bindings} connect Ports inside an ActorClass.

Example:

\begin{table}
\caption{Actor Class Example}
\begin{tabular}{|l|l|l|}
\hline
 \textbf{Graphical Notation} & \textbf{Textual Notation} \\ \hline
 \includegraphics{images/040-ActorClass.png} & \includegraphics{images/040-ActorClassExampleTextualNotation.png} \\ \hline
 \end{tabular}
 \end{table}
 
% <table title="Actor Class Example" frame="box" border="2" cellpadding="3" cellspacing="0" >
	% <tr>
		% <td align="center">*Graphical Notation*</td>
		% <td align="center">*Textual Notation*</td>
	% </tr>
	% <tr>
		% <td>!images/040-ActorClass.png!</td>
		% <td>!images/040-ActorClassExampleTextualNotation.png!</td>
	% </tr>
% </table> 

\begin{itemize}
\item \textbf{ActorClass1} contains two ActorReferences (of ActorClass2 and ActorClass3)
\item \textit{port1} is a \textbf{External End Port}. Since it connects external Actors with the behavior of the ActorClass, it is defined in the \textbf{Interface} section and the \textbf{Structure} section of the ActorClass.
\item \textit{port2} and \textit{port3} are \textbf{Internal End Ports} and can only be connected to the ports of contained ActorReferences. Internal End Ports connect the Behavior of an ActorClass with its contained ActorReferences.
\item \textit{port4} is a relay port and connects external Actors to contained ActorReferences. This port can not be accessed by the behavior of the ActorClass.
\item \textit{port5} through \textit{port9} are Ports of contained ActorReferences. \textit{port8} and \textit{port9} can communicate without interference with the containing ActorClass.
\item \textbf{Bindings} can connect ports of the ActorClass and its contained ActorReferences. 
\end{itemize}

\subsubsection{Attributes}

Attributes are part of the Structure of an ActorClass. They can be of a PrimitiveType or a DataClass.

Example:

\includegraphics{images/040-ActorClassAttributes.png}
% !images/040-ActorClassAttributes.png!

\subsubsection{Operations}

Operations are part of the Behavior of an ActorClass.  Arguments and return values can be of a PrimitiveType or a DataClass. DataClasses can be passed by value (implicit) or by reference (keyword \textbf{ref}).

Example:

\includegraphics{images/040-ActorClassOperations.png}
% !images/040-ActorClassOperations.png!

\section{Protocols}

\subsection{Description}

A ProtocolClass defines a set of incoming and outgoing messages that can be exchanged between two ports.
The exact semantics of a message is defined by the execution model.

\subsection{Motivation}

\begin{itemize}
\item ProtocolClasses provide a reusable interface specification for ports
\item ProtocolClasses can optionally specify valid message exchange sequences
\end{itemize}

\subsection{Notation}

ProtocolClasses have only textual notation. 
The example defines a ProtocolClass with 2 incoming and two outgoing messages. Messages can have data attached. The data can be of a primitive type (e.g. int32, float64, ...) or a DataClass.

\includegraphics{images/040-ProtocolClassTextualNotation.png}
!images/040-ProtocolClassTextualNotation.png!

\section{Ports}

\subsection{Description}

Ports are the only interfaces of actors. A port has always a protocol assigned. 
Service Access Points (SAP) and Service Provision Points (SPP) are specialized ports that are used to define layering.

\subsection{Motivation}

\begin{itemize}
\item Ports decouple interface definition (Protocols) from interface usage
\item Ports decouple the logical interface from the transport 
\end{itemize}

\subsection{Notation}

\subsubsection{Class Ports}

These symbols can only appear on the border of an actor class symbol. 

Ports that define an external interface of the ActorClass, are defined in the \textit{Interface}. Ports that define an internal interface are defined in the \textit{Structure} (e.g. internal ports).
\begin{itemize}
\item \textbf{External End Ports} are defined in the Interface and the Structure
\item \textbf{Internal End Ports} are only defined in the Structure
\item \textbf{Relay Ports} are only defined in the Interface
\item \textbf{End Ports} are always connected to the internal behavior of the ActorClass
\item \textbf{Replicated Ports} can be defined with a fixed replication factor ( e.g. \textit{Port port18 [ 5 ]: ProtocolClass1} ) or a variable replication factor (e.g. \textit{Port port18[ * ]: ProtocolClass1} )
\end{itemize}

\begin{table}
\caption{Class Port Notation}
\begin{tabular}{|l|l|l|}
\hline
 \textbf{Element} & \textbf{Graphical Notation} & \textbf{Textual Notation} \\ \hline
 Class End Port & \includegraphics{images/040-ClassEndPort.png} & \\ \hline
 Class End Port & \includegraphics{images/040-ConjugatedClassEndPort.png} & \\ \hline
 Class Relay Port & \includegraphics{images/040-ClassRelayPort.png} & \\ \hline
 Conjugated Class Relay Port & \includegraphics{images/040-ConjugatedClassRelayPort.png} & \\ \hline
 Replicated Class End Port & \includegraphics{images/040-ReplicatedClassEndPort.png} & \\ \hline
 Conjugated Replicated Class End Port & \includegraphics{images/040-ConjugatedReplicatedClassEndPort.png} & \\ \hline
 Replicated Class Relay Port & \includegraphics{images/040-ReplicatedClassRelayPort.png} & \\ \hline
 Conjugated Replicated Class Relay Port & \includegraphics{images/040-ConjugatedReplicatedClassRelayPort.png} & \\ \hline
\end{tabular}
\end{table}

% <table title="Class Port Notation" frame="box" border="2" cellpadding="3" cellspacing="0">
	% <tr>
		% <td align="center">*Element*</td>
		% <td align="center" width="15%">*Graphical Notation*</td>
		% <td align="center">*Textual Notation*</td>
	% </tr>
	% <tr>
		% <td>Class End Port</td>
		% <td align="center">!images/040-ClassEndPort.png!</td>
		% <td>
			% *External Class End Port:*
			% !images/040-ClassEndPortTextual.png!
			% *Internal Class End Port:*
			% !images/040-ClassEndPortInternalTextual.png!
		% </td>
	% </tr>
	% <tr>
		% <td>Conjugated Class End Port</td>
		% <td align="center">!images/040-ConjugatedClassEndPort.png!</td>
		% <td>
			% *External Conjugated Class End Port:*
			% !images/040-ConjugatedClassEndPortTextual.png!
			% *Internal Conjugated Class End Port:*
			% !images/040-ConjugatedClassEndPortInternalTextual.png!
		% </td>
	% </tr>
	% <tr>
		% <td>Class Relay Port</td>
		% <td align="center">!images/040-ClassRelayPort.png!</td>
		% <td>
			% !images/040-ClassRelayPortTextual.png!
		% </td>
	% </tr>
	% <tr>
		% <td>Conjugated Class Relay Port</td>
		% <td align="center">!images/040-ConjugatedClassRelayPort.png!</td>
		% <td>
			% !images/040-ConjugatedClassRelayPortTextual.png!
		% </td>
	% </tr>
	% <tr>
		% <td>Replicated Class End Port</td>
		% <td align="center">!images/040-ReplicatedClassEndPort.png!</td>
		% <td>
			% *External Replicated Class End Port:*
			% !images/040-ReplicatedClassEndPortTextual.png!
			% *Internal Replicated Class End Port:*
			% !images/040-ReplicatedClassEndPortInternalTextual.png!
		% </td>
	% </tr>
	% <tr>
		% <td>Conjugated Replicated Class End Port</td>
		% <td align="center">!images/040-ConjugatedReplicatedClassEndPort.png!</td>
		% <td>
			% *External Conjugated Replicated Class End Port:*
			% !images/040-ConjugatedReplicatedClassEndPortTextual.png!
			% *Internal Conjugated Replicated Class End Port:*
			% !images/040-ConjugatedReplicatedClassEndPortInternalTextual.png!
		% </td>
	% </tr>
	% <tr>
		% <td>Replicated Class Relay Port</td>
		% <td align="center">!images/040-ReplicatedClassRelayPort.png!</td>
		% <td>
			% !images/040-ReplicatedClassRelayPortTextual.png!
		% </td>
	% </tr>
	% <tr>
		% <td>Conjugated Replicated Class Relay Port</td>
		% <td align="center">!images/040-ConjugatedReplicatedClassRelayPort.png!</td>
		% <td>
			% !images/040-ConjugatedReplicatedClassRelayPortTextual.png!
		% </td>
	% </tr>
% </table>

\subsubsection{Reference Ports}

These symbols can only appear on the border of an ActorReference symbol. Since the type of port is defined in the ActorClass, no textual notation for the Reference Ports exists.

\begin{table}
\caption{Title}
\begin{tabular}{|c|c|c|}
\hline
 \textbf{Element} & \textbf{Graphical Notation} & \textbf{Textual Notation} \\ \hline
 Reference Port & \includegraphics{images/040-ReferencePort.png} & \textit{implicit} \\ \hline
 Conjugated Reference Port & \includegraphics{images/040-ConjugatedReferencePort.png} & \textit{implicit} \\ \hline
 Replicated Reference Port & \includegraphics{images/040-ReplicatedReferencePort.png} & \textit{implicit} \\ \hline
 Conjugated Replicated Reference Port & \includegraphics{images/040-ConjugatedReplicatedReferencePort.png} & \textit{implicit} \\ \hline
\end{tabular}
\end{table}


% <table title="Title" frame="box" border="2" cellpadding="3" cellspacing="0">
	% <tr>
		% <td align="center">*Element*</td>
		% <td align="center">*Graphical Notation*</td>
		% <td align="center">*Textual Notation*</td>
	% </tr>
	% <tr>
		% <td>Reference Port</td>
		% <td align="center">!images/040-ReferencePort.png!</td>
		% <td align="center">_implicit_</td>
	% </tr>
	% <tr>
		% <td>Conjugated Reference Port</td>
		% <td align="center">!images/040-ConjugatedReferencePort.png!</td>
		% <td align="center">_implicit_</td>
	% </tr>
	% <tr>
		% <td>Replicated Reference Port</td>
		% <td align="center">!images/040-ReplicatedReferencePort.png!</td>
		% <td align="center">_implicit_</td>
	% </tr>
	% <tr>
		% <td>Conjugated Replicated Reference Port</td>
		% <td align="center">!images/040-ConjugatedReplicatedReferencePort.png!</td>
		% <td align="center">_implicit_</td>
	% </tr>
% </table>

\section{DataClass}

\subsection{Description}

The DataClass enables the modeling of hierarchical complex datatypes and operations on them. The DataClass is the equivalent to a Class in languages like Java or C++, but has less features. The content of a DataClass can always be sent via message between actors (defined as message data in ProtocolClass).

\subsection{Notation}
  
Example: DataClass using PrimitiveTypes

\includegraphics{images/040-DataClass1.png}
% !images/040-DataClass1.png!

Example: DataClass using other DataClasses:

\includegraphics{images/040-DataClass2.png}
% !images/040-DataClass2.png!

\section{Layering}

\subsection{Description}

In addition to the Actor containment hierarchies, Layering provides another method to hierarchically structure a software system. Layering and actor hierarchies with port to port connections can be mixed on every level of granularity.
\begin{enumerate}
\item an ActorClass can define a Service Provision Point (SPP) to publish a specific service, defined by a ProtocolClass
\item an ActorClass can define a Service Access Point (SAP) if it needs a service, defined by a ProtocolClass
\item for a given Actor hierarchy, a LayerConnection defines which SAP will be satisfied by (connected to) which SPP
\end{enumerate}

\subsection{Notation}

\begin{table}
\begin{tabular}{|c|c|c|}
\hline
 \textbf{Description} & \textbf{Graphical Notation} & \textbf{Textual Notation} \\ \hline
 The Layer Connections in this model define which services are provided by the \textit{ServiceLayer} (\textit{digitalIO} and \textit{timer}) & \includegraphics{images/040-LayeringModel.png} & \includegraphics{images/040-LayeringModelTextual.png}  \\ \hline
 The implementation of the services (SPPs) can be delegated to sub actors. In this case the actor \textit{ServiceLayer} relays (delegates) the implementation services \textit{digitalIO} and \textit{timer} to sub actors & \includegraphics{images/040-LayeringServiceLayer.png} & \includegraphics{images/040-LayeringServiceLayerTextual.png} \\ \hline
 Every Actor inside the \textit{ApplicationLayer} that contains an SAP with the same Protocol as \textit{timer} or \textit{digitalIO} will be connected to the specified SPP & \includegraphics{images/040-LayeringApplicationLayer.png} & \includegraphics{images/040-LayeringApplicationLayerTextual.png} \\ \hline
\end{tabular}
\end{table}

% <table title="Title" frame="box" border="2" cellpadding="3" cellspacing="0">
	% <tr>
		% <td align="center">*Description*</td>
		% <td align="center">*Graphical Notation*</td>
		% <td align="center">*Textual Notation*</td>
	% </tr>
	% <tr>
		% <td>The Layer Connections in this model define which services are provided by the _ServiceLayer_  (_digitalIO_ and _timer_)</td>
		% <td>!images/040-LayeringModel.png!</td>
		% <td>!images/040-LayeringModelTextual.png!</td>
	% </tr>
	% <tr>
		% <td>The implementation of the services (SPPs) can be delegated to sub actors. In this case the actor _ServiceLayer_ relays (delegates) the implementation services _digitalIO_ and _timer_ to sub actors</td>
		% <td>!images/040-LayeringServiceLayer.png!</td>
		% <td>!images/040-LayeringServiceLayerTextual.png!</td>
	% </tr>
	% <tr>
		% <td>Every Actor inside the _ApplicationLayer_ that contains an SAP with the same Protocol as _timer_ or _digitalIO_ will be connected to the specified SPP</td>
		% <td>!images/040-LayeringApplicationLayer.png!</td>
		% <td>!images/040-LayeringApplicationLayerTextual.png!</td>
	% </tr>
% </table>

\section{Finite State Machines}

\subsection{Description}

Definition from "Wikipedia":http://en.wikipedia.org/wiki/Finite-state\_machine:

\begin{verbatim}
A finite-state machine (FSM) or finite-state automaton (plural: automata), or simply a state machine, is a mathematical model used to design computer programs and digital logic circuits. It is conceived as an abstract machine that can be in one of a finite number of states. The machine is in only one state at a time; the state it is in at any given time is called the current state. It can change from one state to another when initiated by a triggering event or condition, this is called a transition. A particular FSM is defined by a list of the possible states it can transition to from each state, and the triggering condition for each transition.

In ROOM each actor class can implement its behavior using a state machine. Events occurring at the end ports of an actor will be forwarded to and processed by the state machine. Events possibly trigger state transitions.
\end{verbatim}

\subsection{Motivation}

For event driven systems a finite state machine is ideal for processing the stream of events. Typically during processing new events are produced which are sent to peer actors.

We distinguish flat and hierarchical state machines.

\subsection{Notation}

\subsubsection{Flat Finite State Machine}

The simpler flat finite state machines are composed of the following elements:

\begin{table}
\caption{Title}
\begin{tabular}{|c|c|c|}
\hline
 \textbf{Description} & \textbf{Graphical Notation} & \textbf{Textual Notation} \\ \hline
 State & \includegraphics{images/040-State.jpg} & \includegraphics{images/040-StateTextual.jpg} \\ \hline
 InitialPoint & \includegraphics{images/040-InitialPoint.jpg} & \textit{implicit} \\ \hline
 TransitionPoint & \includegraphics{images/040-TransitionPoint.jpg} & \includegraphics{images/040-TransitionPointTextual.jpg} \\ \hline
 ChoicePoint & \includegraphics{images/040-ChoicePoint.jpg} & \includegraphics{images/040-ChoicePointTextual.jpg} \\ \hline
 Initial Transition & \includegraphics{images/040-InitialTransition.jpg} & \includegraphics{images/040-InitialTransitionTextual.jpg} \\ \hline
 Triggered Transition & \includegraphics{images/040-TriggeredTransition.jpg} & \includegraphics{images/040-TriggeredTransitionTextual.jpg} \\ \hline
\end{tabular}
\end{table}
% <table title="Title" frame="box" border="2" cellpadding="3" cellspacing="0">
	% <tr>
		% <td align="center">*Description*</td>
		% <td align="center">*Graphical Notation*</td>
		% <td align="center">*Textual Notation*</td>
	% </tr>
	% <tr>
		% <td>State</td>
		% <td>!images/040-State.jpg!</td>
		% <td>!images/040-StateTextual.jpg!</td>
	% </tr>
	% <tr>
		% <td>InitialPoint</td>
		% <td>!images/040-InitialPoint.jpg!</td>
		% <td>_implicit_</td>
	% </tr>
	% <tr>
		% <td>TransitionPoint</td>
		% <td>!images/040-TransitionPoint.jpg!</td>
		% <td>!images/040-TransitionPointTextual.jpg!</td>
	% </tr>
	% <tr>
		% <td>ChoicePoint</td>
		% <td>!images/040-ChoicePoint.jpg!</td>
		% <td>!images/040-ChoicePointTextual.jpg!</td>
	% </tr>
	% <tr>
		% <td>Initial Transition</td>
		% <td>!images/040-InitialTransition.jpg!</td>
		% <td>!images/040-InitialTransitionTextual.jpg!</td>
	% </tr>
	% <tr>
		% <td>Triggered Transition</td>
		% <td>!images/040-TriggeredTransition.jpg!</td>
		% <td>!images/040-TriggeredTransitionTextual.jpg!</td>
	% </tr>
% </table>


\subsubsection{Hierarchical Finite State Machine}

The hierarchical finite state machine adds the notion of a sub state machine nested in a state.
A few modeling elements are added to the set listed above:

\begin{table}
\caption{Title}
\begin{tabular}{|c|c|c|}
\hline
 \textbf{Description} & \textbf{Graphical Notation} & \textbf{Textual Notation} \\ \hline
 State with sub state machine &  & \includegraphics{images/040-StateWithSubFSMTextual.jpg} \\ \hline
 Entry Point & & \includegraphics{images/040-EntryPointTextual.jpg} \\ \hline
 Exit Point & & \includegraphics{images/040-ExitPointTextual.jpg} \\ \hline
\end{tabular}
\end{table}

% <table title="Title" frame="box" border="2" cellpadding="3" cellspacing="0">
	% <tr>
		% <td align="center">*Description*</td>
		% <td align="center">*Graphical Notation*</td>
		% <td align="center">*Textual Notation*</td>
	% </tr>
	% <tr>
		% <td>State with sub state machine</td>
		% <td>Parent State
		% !images/040-StateWithSubFSM.jpg!
		% Sub state machine
		% !images/040-SubFSM.jpg!</td>
		% <td>!images/040-StateWithSubFSMTextual.jpg!</td>
	% </tr>
	% <tr>
		% <td>Entry Point</td>
		% <td>In sub state machine
		% !images/040-EntryPoint.jpg!
		% On parent state
		% !images/040-EntryPointRef.jpg!</td>
		% <td>!images/040-EntryPointTextual.jpg!</td>
	% </tr>
	% <tr>
		% <td>Exit Point</td>
		% <td>In sub state machine
		% !images/040-ExitPoint.jpg!
		% On parent state
		% !images/040-ExitPointRef.jpg!</td>
		% <td>!images/040-ExitPointTextual.jpg!</td>
	% </tr>
% </table>


\subsection{Examples}

\subsubsection{Example of a flat finite state machine:}

% !images/040-FlatFSM.jpg!
\includegraphics{images/040-FlatFSM.jpg}

\subsubsection{Example of a hierarchical finite state machine:}

Top level
% !images/040-HierarchicalFSMTop.jpg!
\includegraphics{images/040-HierarchicalFSMTop.jpg}

Sub state machine of Initializing
% !images/040-HierarchicalFSMInitializing.jpg!
\includegraphics{images/040-HierarchicalFSMInitializing.jpg}

Sub state machine of Running
% !images/040-HierarchicalFSMRunning.jpg!
\includegraphics{images/040-HierarchicalFSMRunning.jpg}
\chapter{eTrice Features}

\section{Codegenerators}

\subsection{Java Generator}

\subsection{C++ Generator}

\subsection{C Generator}


\include{060-codegenerators}
\include{070-runtimes}
\section{\eTrice{} Models and Their Relations}

\eTrice{} comprises several models:

\begin{itemize}
\item the ROOM model (*.room) -- defines model classes and the logical structure of the model
\item the Config model (*.config) -- defines configuration values for attributes
\item the Physical model (*.etphys) -- defines the structure and properties of the physical system
\item the Mapping model (*.etmap) -- defines a mapping from logical elements to physical elements
\end{itemize}

In the following diagram the models and their relations are depicted. The meaning of the arrows is: 
uses/references.

\includegraphics[scale=0.4]{images/080-models.jpg}

In the following sections we will describe those models with emphasis of their cross relations.

\subsection{The ROOM Model}

The ROOM model defines \room{DataClass}es, \room{ProtocolClass}es, \room{ActorClass}es, \room{SubSystemClass}es and \room{LogicalSystem}s.
Thereby the three latter form a hierarchy. The \room{LogicalSystem} is the top level element of the structure. 
It contains references to \room{SubSystemClass} elements. The \room{SubSystemClass} in turn contains 
references to \room{ActorClass} elements which again contain (recursively) references to 
\room{ActorClass} elements. The complete structural hierarchy implies a tree which has the 
\room{LogicalSystem} as root and where each reference stands for a new node with possibly further 
branches.

Let's consider a simple example. It doesn't implement anything meaningful and completely omits behavioral and 
other aspects.

\includegraphics{images/080-room.jpg}

When a \room{LogicalSystem} is instantiated then recursively all of the contained referenced elements are 
instantiated as instances of the corresponding class. Thus the instance tree of the above example looks like 
in figure \ref{fig:instance_tree} (the third line in the white boxes shows some mapping information,
see section \ref{sec:mapping_model} \nameref{sec:mapping_model}):

\begin{figure}
\includegraphics[scale=0.45]{images/080-instances.jpg}
\caption{Instances of a ROOM system}
\label{fig:instance_tree}
\end{figure}

Here is a long example of a ROOM file (not complete):

\begin{lstlisting}[language=ROOM]
RoomModel trafficlight.example {

	import room.basic.types.* from "../../../org.eclipse.etrice.modellib.c/model/Types.room"

	import room.basic.service.timing.* from "../../../org.eclipse.etrice.modellib.c/model/TimingService.room"

	import room.basic.service.tcp.* from "../../../org.eclipse.etrice.modellib.c/model/TcpService.room"

	LogicalSystem LSTraffic {
		SubSystemRef main: SSTraffic
	}
	
	SubSystemClass SSTraffic ["Subsystem of Trafficlight Example Application. The Subsystem contains all Actors of the application."] {
		ActorRef application: TrafficlightExampleApplication ["reference to application"]
		ActorRef TimingService: ATimingService ["reference to timing service"]
		LayerConnection ref application satisfied_by TimingService.timer 
	
		LogicalThread dflt_thread
	}

	ActorClass TrafficlightExampleApplication ["Toplevel Actor of the Trafficlight Example Application."]{
		Structure {
			
			ActorRef light1: TrafficLight ["first traffic light"]
			ActorRef light2: TrafficLight ["second traffic light"]
			ActorRef controller: TrafficController ["controller for coordination of the traffic lights"]
			Binding controller.light1 and light1.controller 
			Binding controller.light2 and light2.controller
		}
		Behavior { }
	}

	ActorClass TrafficController ["The TrafficController coordinates two traffic lights (directions)."] {
		Interface {
			conjugated Port light1: PTrafficLight ["port to control traffic light 1"]
			conjugated Port light2: PTrafficLight ["port to control traffic light 2"]
		}
		Structure {
			usercode1 {
				"#include \"platform/etTcpSockets.h\""
			}
			external Port light1 
			external Port light2 
			SAP timeout: PTimer
		}
		Behavior {
			Operation TrafficController() {
				"etInitSockets();"
			}
			Operation ~TrafficController() {
				"etCleanupSockets();"
			}
			StateMachine {
				Transition init: initial -> Idle { }
				Transition tr0: Idle -> SwitchToLight1GreenForCars {
					triggers {
						<timeout: timeout>
					}
				}
				Transition tr1: SwitchToLight1GreenForCars -> state0 {
					triggers {
						<greenForCarDone: light1>
					}
				}
				Transition tr2: SwitchToLight1GreenForCars -> state1 {
					triggers {
						<greenForPedDone: light2>
					}
				}
				Transition tr3: state1 -> Light1GreenForCars {
					triggers {
						<greenForCarDone: light1>
					}
				}
				Transition tr4: state0 -> Light1GreenForCars {
					triggers {
						<greenForPedDone: light2>
					}
				}
				Transition tr5: Light1GreenForCars -> SwitchToLight2GreenForCars {
					triggers {
						<timeout: timeout>
					}
				}
				Transition tr6: SwitchToLight2GreenForCars -> state2 {
					triggers {
						<greenForPedDone: light1>
					}
				}
				Transition tr7: SwitchToLight2GreenForCars -> state3 {
					triggers {
						<greenForCarDone: light2>
					}
				}
				Transition tr8: state2 -> Light2GreenForCars {
					triggers {
						<greenForCarDone: light2>
					}
				}
				Transition tr9: state3 -> Light2GreenForCars {
					triggers {
						<greenForPedDone: light1>
					}
				}
				Transition tr10: Light2GreenForCars -> SwitchToLight1GreenForCars {
					triggers {
						<timeout: timeout>
					}
				}
				State Idle {
					entry {
						"timeout.startTimeout(3000);"
					}
				}
				State Light1GreenForCars {
					entry {
						"timeout.startTimeout(10000);"
					}
				}
				State SwitchToLight1GreenForCars {
					entry {
						"light1.greenForCar();"
						"light2.greenForPed();"
					}
				}
				State state0
				State state1
				State SwitchToLight2GreenForCars {
					entry {
						"light1.greenForPed();"
						"light2.greenForCar();"
					}
				}
				State state2
				State state3
				State Light2GreenForCars {
					entry {
						"timeout.startTimeout(10000);"
					}
				}
			}
		}
	}
\end{lstlisting}

\subsection{The Config Model}

Once we have the ROOM class model we can configure values using the Config model. This can be done on the 
class level and/or on the instance level. Values defined for class attributes are used for all instances 
unless there is an instance value configured for the same attribute.

\includegraphics{images/080-config.jpg}

\begin{lstlisting}[language=Config]
ConfigModel de.protos.automation.MachineElementsTest.config {
	import de.protos.automation.MachineElementsTest.* from "MachineElementsTest.room"
	
	ActorInstanceConfig MachineElementsTestSystem/ioServiceTest/test/cylinder {
		Attr channelOutPos0 = 0
		Attr channelOutPos1 = 1
		Attr channelInPos0 = 0
		Attr channelInPos1 = 1
		Attr errorTimeoutConfig = 3000
	}
	
	ActorInstanceConfig MachineElementsTestSystem/ioServiceTest/test/cylinderSim {
		Attr channelOutPos0 = 0
		Attr channelOutPos1 = 1
		Attr channelInPos0 = 0
		Attr channelInPos1 = 1

		Attr minPosition = 0
		Attr maxPosition = 100
		Attr currentPosition = 0
		Attr stepWidth = 10
	}
}\end{lstlisting}

\subsection{The Physical Model}

The physical model defines the physical resources onto which the logical system will be deployed. It is 
possible to define runtime classes which (currently) only define the overall execution model of the 
platform.

\includegraphics{images/080-runtimes.jpg}

The \room{PhysicalSystem} is composed of \room{NodeRef}erences which are instances
of \room{NodeClass}es. Each \room{NodeClass} os referencing a 
\room{RuntimeClass} and is defining \room{Threads}.

\includegraphics{images/080-phys.jpg}

\begin{lstlisting}[language=etPhys]
PhysicalModel cGenRef {

	PhysicalSystem Sys {
		NodeRef node1: PC
		NodeRef node2: PC
	}
	
	NodeClass PC {
		runtime = PC
		priomin = 1
		priomax = 5
		
		DefaultThread PhysicalThread1 {
			execmode = blocked
			prio = 5
			stacksize = 1024
			msgblocksize = 64
			msgpoolsize = 256
		}
		
		Thread PhysicalThread2 {
			execmode = blocked
			prio = 5
			stacksize = 1024
			msgblocksize = 64
			msgpoolsize = 256
		}
	}
	
	RuntimeClass PC {
		model = multiThreaded
	}
}
\end{lstlisting}

\subsection{The Mapping Model}
\label{sec:mapping_model}

The last model finally combines all this information by mapping logical to physical entities.

\includegraphics{images/080-map.jpg}

\begin{lstlisting}[language=etMap]
MappingModel cgenRef {
	
	import cGenRef from "cGenRef.room"
	import cGenRef from "cGenRef.etphys"
	
	Mapping cGenRef.LS -> cGenRef.Sys {
		SubSystemMapping sys1 -> node1 {
			ThreadMapping dflt_thread -> PhysicalThread1
			ThreadMapping other_thread -> PhysicalThread2
		}
		SubSystemMapping sys2 -> node2 {
			ThreadMapping dflt_thread -> PhysicalThread1
			ThreadMapping other_thread -> PhysicalThread2
		}
	}
}
\end{lstlisting}

The result of the mapping is also depicted in above tree diagram (figure \ref{fig:instance_tree})
of the instances. All actor instances (the white boxes) are mapped to a node and a thread running on this node
(shown as @\textit{node} : \textit{thread}).


\section{Architecture}

The basic components of \eTrice{} are depicted in the following diagram.

\includegraphics[scale=0.5]{images/200-components.jpg}

Additional to that the \eTrice{} project comprises runtime libraries and unit tests which are treated in 
subsequent sections.

\subsection{Editor and Generator Components}

\begin{itemize}
\item core

\begin{itemize}
\item core.common is an Xtext based language which serves as a base for other \eTrice{} languages.
It consists of the plug-ins
\texttt{org.eclipse.etrice.core.common} and
\texttt{org.eclipse.etrice.core.common.ui}.
The base grammar defines recurring items like numbers with literals, annotations and the like.
\item core.fsm is an Xtext based language that defines state machines in an abstract way.
It consists of the plug-ins
\texttt{org.eclipse.etrice.core.fsm} and
\texttt{org.eclipse.etrice.core.fsm.ui}.
The FSM language is abstract and has to be embedded in a model that defines containers for the state machine
with interface items (e.g. ROOM ports or Franca interfaces) and messages.
The ROOM grammar of \eTrice{} is derived from this grammar.
\item core.room is an Xtext based language called ROOM. It consists of the plug-ins
\texttt{org.eclipse.etrice.core.room} and
\texttt{org.eclipse.etrice.core.room.ui}. ROOM is the basic modeling language of \eTrice{}.
\item core.config is an Xtext based language called Config. It consists of the plug-ins
\texttt{org.eclipse.etrice.core.config} and
\texttt{org.eclipse.etrice.core.config.ui}. Config is a language designed for the data configuration of model 
\item core.etphys is an Xtext based language called etPhys. It consists of the plug-ins
\texttt{org.eclipse.etrice.core.etphys} and
\texttt{org.eclipse.etrice.core.etphys.ui}. etPhys is a language designed for the description of physical systems
onto which the logical ROOM systems are deployed.
\item core.etmap is an Xtext based language called etMap. It consists of the plug-ins
\texttt{org.eclipse.etrice.core.etmap} and
\texttt{org.eclipse.etrice.core.etmap.ui}. etMap is a language designed for the mapping of logical
to physical systems.
\item core.genmodel.fsm is an EMF based aggregation layer for finite state machines. It consists of the plugin 
\texttt{org.eclipse.etrice.core.genmodel.fsm}. A \texttt{ModelComponent} can be transformed into a \texttt{ExpandedModelComponent} which is an
explicit version of the state machine with all the inherited items contained.
\item core.genmodel is an EMF based aggregation layer for Room models. It consists of the plugin 
\texttt{org.eclipse.etrice.core.genmodel}. A Room model can be transformed into a genmodel which allows 
easy access to implicit relations of the Room model.
\end{itemize}

\item ui
\begin{itemize}
\item textual
\begin{itemize}

\item fsm.ui is the ui counterpart of core.fsm.  It consists of the plug-in 
\texttt{org.eclipse.etrice.core.fsm.ui}. This plug-in realizes IDE concepts like content assist, error 
markers and navigation by hyper links for the FSM language.
\item room.ui is the ui counterpart of core.room.  It consists of the plug-in 
\texttt{org.eclipse.etrice.core.room.ui}. This plug-in realizes IDE concepts like content assist, error 
markers and navigation by hyper links for the Room language.
\item config.ui is the ui counterpart of core.config.  It consists of the plug-in 
\texttt{org.eclipse.etrice.core.config.ui}. This plug-in realizes IDE concepts like content assist, error 
markers and navigation by hyper links for the Config language.
\item etphys.ui is the ui counterpart of core.etphys.  It consists of the plug-in 
\texttt{org.eclipse.etrice.core.etphys.ui}. This plug-in realizes IDE concepts like content assist, error 
markers and navigation by hyper links for the etPhys language.
\item etmap.ui is the ui counterpart of core.etmap.  It consists of the plug-in 
\texttt{org.eclipse.etrice.core.etmap.ui}. This plug-in realizes IDE concepts like content assist, error 
markers and navigation by hyper links for the etPhys language.
\end{itemize}

\item graphical
\begin{itemize}
\item ui.common.base is a set of common code for the diagram editors. It consists of the plug-in 
\texttt{org.eclipse.etrice.ui.common.base}. It depends only on the FSM part but not on ROOM.
\item ui.common is a set of common code for the two diagram editors. It consists of the plug-in 
\texttt{org.eclipse.etrice.ui.common}.
\item ui.commands encapsulates some commands related to the navigation between \eTrice{} editors. It consists 
of the plug-in \texttt{org.eclipse.etrice.ui.commands}.
\item ui.structure is the Graphiti based editor for the Actor structure. It consists of the plug-in 
\texttt{org.eclipse.etrice.ui.structure}.
\item ui.behavior.fsm is implementing the major part for the graphical state machine editor. It consists of the plug-in 
\texttt{org.eclipse.etrice.ui.behavior.fsm}. All property dialogs are handled in an abstract way
using a factory.
\item ui.behavior is the Graphiti based editor for the Actor behavior. It consists of the plug-in 
\texttt{org.eclipse.etrice.ui.behavior}. It utilizes the ui.behavior.fsm and provides concrete property dialogs.
\end{itemize}
\end{itemize}

\item generators
\begin{itemize}
\item generator.fsm is a set of general classes and language independent parts of all generators. It consists 
of the plug-in \textit{org.eclipse.etrice.generator.fsm}. It depends only on FSM but not on ROOM.
\item generator is a set of general classes and language independent parts of all generators. It consists 
of the plug-in \textit{org.eclipse.etrice.generator}.
\item generator.c is the generator for the ANSI-C target language. It consists of the plug-in 
\texttt{org.eclipse.etrice.generator.c}.
\item generator.cpp is the generator for the C++ target language. It consists of the plug-in 
\texttt{org.eclipse.etrice.generator.cpp}.
\item generator.java is the generator for the Java target language. It consists of the plug-in 
\texttt{org.eclipse.etrice.generator.java}.
\item generator.doc is the generator for the model documentation. It consists of the plug-in 
\texttt{org.eclipse.etrice.generator.doc}.
\end{itemize}
\end{itemize}

\subsection{The Abstract Finite State Machine Concept}

\eTrice{} comes with an easy to re-use concept of hierarchical finite state machines (FSM for short).
A powerful inheritance concept is used and there is also state machine validation based on
semantic rules for messages and abstract execution available.

State machines are an integral part of the ROOM language. But they can also be used independently from that using

\begin{itemize}

\item for the model part
\begin{itemize}
\item \texttt{org.eclipse.etrice.core.common}
\item \texttt{org.eclipse.etrice.core.fsm}
\item \texttt{org.eclipse.etrice.core.genmodel.fsm}
\end{itemize}

\item graphical state machine editor
\begin{itemize}
\item \texttt{org.eclipse.etrice.core.common.ui}
\item \texttt{org.eclipse.etrice.core.fsm.ui}
\item \texttt{org.eclipse.etrice.core.ui.common.base}
\item \texttt{org.eclipse.etrice.core.ui.common}
\end{itemize}

\item base classes for code generation
\begin{itemize}
\item \texttt{org.eclipse.etrice.generator.fsm}
\end{itemize}

\item validation by abstract execution
\begin{itemize}
\item \texttt{org.eclipse.etrice.abstractexec.behavior}
\end{itemize}

\end{itemize}

The first three parts have to be used by concrete implementations that implement the abstract interface.
\eTrice{} itself uses the abstract FSMs in exactly this way.

\subsubsection*{Extending the FSM Model}

The \eTrice{} FSM model has to be embedded in a model that introduces components, interfaces and messages.
We recommend to use a new Xtext language with a grammar derived from the FSM grammar.
This grammar has to specify a component derived from the \texttt{ModelComponent} of the FSM model.
It further has to introduce concrete realizations of interface items derived from \texttt{AbstractInterfaceItem}.
The interface item is an object contained in a component that has a name (role) and holds a reference to some kind of interface of the
component (like a Franca interface or a ROOM protocol).
Finally a concrete message type derived from an \texttt{EObject} has to be defined. The minimal requirement is that this concrete message
has an attribute called 'name' of type String.

The minimal interface to be implemented consists of
\begin{itemize}
	\item for the concrete interface item
	\begin{itemize}
		\item \texttt{EList<EObject> getAllIncomingAbstractMessages()}
		\item \texttt{EList<EObject> getAllOutgoingAbstractMessages()}
		\item \texttt{ProtocolSemantics getSemantics()}
	\end{itemize}
	\item for the concrete model component
	\begin{itemize}
		\item \texttt{EList<AbstractInterfaceItem> getAbstractInterfaceItems} -- the interface items contained in this model component
		\item \texttt{EList<AbstractInterfaceItem> getAllAbstractInterfaceItems} -- all interface items including inherited ones
		\item \texttt{String getComponentName()} -- should return the name of the model component
	\end{itemize}
\end{itemize}

\subsubsection*{Extending the State Machine Editor}

The concrete state machine editor minimally needs to define
\begin{itemize}
\item the editor class itself by deriving it from the \texttt{AbstractFSMEditor}
\item a diagram type provider (which may derive from \texttt{AbstractDiagramTypeProvider})
\item a Google Guice module with bindings for
	\begin{itemize}
	\item \texttt{IFSMDialogFactory}
	\item \texttt{DiagramAccessBase}
	\item \texttt{IBehaviorQuickfixProvider}
	\item \texttt{IResourceSetProvider}
	\end{itemize}
\item concrete implementations of all property dialogs the \texttt{IFSMDialogFactory} produces
\end{itemize}


\subsection{Runtimes}

Currently \eTrice{} ships with a C and a Java runtime. The C++ runtime is still a prototype.
The runtimes are libraries written in the target 
language against which the generated code is compiled.

\subsection{Unit Tests}

Most plug-ins and other parts of the code have related unit tests.

\section{Component Overview}

\subsection{Room Language Overview}

We assume that the reader is familiar with the Xtext concepts. So we concentrate on the details of our 
implementation that are worth to be pointed out.

\subsubsection*{Model Tweaks}

All language EMF models of \eTrice{} are inferred from their respective grammar.
However, this powerful mechanism has to be tweaked in some places.

In order to do so post processors are added that are invoked by the Xtext framework on language generation.
This is done for the FSM language by \textit{/org.eclipse.etrice.core.fsm/src/org/eclipse/etrice/core/fsm/postprocessing/ImplPostprocessor.xtend}.

The following parts of the model are changed or added:
\begin{itemize}
\item an operation \texttt{getName} is added to the \texttt{State} class
\item an operation \texttt{getName} is added to the \texttt{StateGraphItem} class
\item an operation \texttt{getSemantics} is added to the \texttt{AbstractInterfaceItem}
\item an operation \texttt{getAllIncomingAbstractMessages} is added to the \texttt{AbstractInterfaceItem}
\item an operation \texttt{getAllOutgoingAbstractMessages} is added to the \texttt{AbstractInterfaceItem}
\item an interface class \texttt{IInterfaceItemOwner} is added
\item an operation \texttt{getAbstractInterfaceItems} is added to the \texttt{AbstractInterfaceItem}
\item an operation \texttt{getAllAbstractInterfaceItems} is added to the \texttt{AbstractInterfaceItem}
\item \texttt{IInterfaceItemOwner} is made a super class of \texttt{ModelComponent}
\end{itemize}
All but the first two items in the list are part of the abstract FSM definition and are used to interface
to the model embedding the FSM language, e.g. ROOM.

For the ROOM language the post processor is
\textit{/org.eclipse.etrice.core.room/src/org/eclipse/etrice/core/RoomPostprocessor.ext}.

The following parts of the model are changed or added:
\begin{itemize}
\item the default \texttt{multiplicity} of the \texttt{Port} is set to 1
\item the operation \texttt{isReplicated} is added to the \texttt{Port}
\item the default \texttt{multiplicity} of the \texttt{ActorRef} is set to 1
\item an operation \texttt{getGeneralProtocol} is added to the \texttt{InterfaceItem}
\item an operation \texttt{getSemantics} is added to the \texttt{InterfaceItem}
\item an operation \texttt{getAllIncomingAbstractMessages} is added to the \texttt{InterfaceItem}
\item an operation \texttt{getAllOutgoingAbstractMessages} is added to the \texttt{InterfaceItem}
\item an operation \texttt{getExternalEndPorts} is added to the \texttt{ActorClass}
\item an operation \texttt{getRelayPorts} is added to the \texttt{ActorClass}
\item an operation \texttt{getImplementedSPPs} is added to the \texttt{ActorClass}
\item an operation \texttt{getActorBase} is added to the \texttt{ActorClass}
\item an operation \texttt{getComponentName} is added to the \texttt{ActorClass}
\item an operation \texttt{getAbstractInterfaceItems} is added to the \texttt{ActorClass}
\item an operation \texttt{getAllAbstractInterfaceItems} is added to the \texttt{ActorClass}
\item an operation \texttt{getStructureClass} is added to the \texttt{ActorContainerRef}
\item an operation \texttt{toString} is added to the \texttt{RefPath}
\item for attribute \texttt{idx} of \texttt{RefSegment} the default is changed to -1
\item an operation \texttt{toString} is added to the \texttt{RefSegment}
\item an operation \texttt{getLiteralValue} is added to the \texttt{EnumLiteral}
\item an operation \texttt{getFullName} is added to the \texttt{EnumLiteral}
\end{itemize}

\subsubsection*{Imports by URI Using Namespaces}

The import mechanism employed is based on URIs. This is configured for one part in the GenerateRoom.mwe2 
model workflow by setting the fragments ImportURIScopingFragment and ImportUriValidator). For the other 
part it is configured in the Guice modules by binding
\begin{itemize}
\item \texttt{PlatformRelativeUriResolver} -- this class tries to convert the import URI into a platform 
relative URI. It also replaces environment variables written in \${} with their respective values.
\item \texttt{ImportedNamespaceAwareLocalScopeProvider} -- this is a standard scope provider which is 
aware of namespaces
\item \texttt{GlobalNonPlatformURIEditorOpener} -- this editor opener tries to convert general URIs into 
platform URIs because editors can only open platform URIs
\item \texttt{ImportAwareHyperlinkHelper} -- turns the URI part of an import into a navigatable hyper link
\end{itemize}

\subsubsection*{Naming}

Two classes provide object names used for link resolution and for labels.
The \texttt{RoomNameProvider} provides frequently used name strings, some of them are hierarchical like 
State paths.
The \texttt{RoomFragmentProvider} serves a more formal purpose since it provides a link between EMF models 
(as used by the diagram editors) and the textual model representation used by Xtext.

\subsubsection*{Helpers}

The \texttt{RoomHelpers} class provides a great deal of static methods that help retrieve frequently used 
information from the model.
Among many, many others
\begin{itemize}
\item \texttt{getAllEndPorts(ActorClass)} - returns a list of all end ports of an actor class including 
inherited ones
\item \texttt{getInheritedActionCode(Transition, ActorClass)} - get the inherited part of a transition's 
action code
\item \texttt{getSignature(Operation)} - returns a string representing the operation signature suited for 
a label
\end{itemize}

\subsubsection*{Validation}

Validation is used from various places. Therefore all validation code is accumulated in the 
@ValidationUtil@ class. All methods are static and many of them return a Result object which contains 
information about the problem detected as well as object and feature as suited for most validation purposes.

\subsection{Config Language Overview}

\subsubsection*{Model Tweaks}

A couple of operations are added to the ConfigModel
\begin{itemize}
\item \texttt{getActorClassConfigs}
\item \texttt{getActorInstanceConfigs}
\item \texttt{getProtocolClassConfigs}
\item \texttt{getSubSystemConfigs}
\end{itemize}

\subsubsection*{Imports by URI Using Namespaces}

Imports are treated like in Room language, section \textit{Imports by URI Using Namespaces}.

\subsubsection*{Util}

A set of static utility methods can be found in the \texttt{ConfigUtil} class.

\subsection{Aggregation Layer Overview}

The \eTrice{} Generator Model (genmodel.fsm and genmodel) serves as an aggregation layer. Its purpose is to allow easy access 
to information which is implicitly contained in the Room model but not simple to retrieve.
Examples of this are the state machine with inherited items or a list of all triggers active at a state in 
the order in which they will be evaluated or the actual peer port of an end port (following bindings 
through relay ports).

The lower level \texttt{FSMGeneratorModelBuilder} takes a \texttt{ModelComponent} and returns a \texttt{ExpandedModelComponent} which
has the inheritance hierarchy of the state machine collapsed into one state machine.
This lower level generator model only depends on general parts and doesn't refer to the ROOM model.

The higher level Generator Model includes the FSM Generator Model.
It is created from a list of Room models by a call of the

\begin{verbatim}createGeneratorModel(List<RoomModel>, boolean)\end{verbatim}

method of the \texttt{GeneratorModelBuilder} class.

The \texttt{Root} object of the resulting Generator Model provides chiefly two things:
\begin{itemize}
\item a tree of instances starting at each \texttt{SubSystem} with representations of each 
\texttt{ActorInstance} and \texttt{PortInstance}
\item for each \texttt{ActorClass} a corresponding \texttt{ExpandedActorClass} with an explicit state 
machine containing all inherited state graph items
\end{itemize}

\subsubsection*{The Instance Model}

The instance model allows easy access to instances including their unique paths and object IDs. Also it is 
possible to get a list of all peer port instances for each port instance without having to bother about 
port and actor replication.

\subsubsection*{The Expanded Model Component}

The expanded model component contains, as already mentioned, the complete state machine of the model component. 
This considerably simplifies the task of state machine generation. Note that the generated code always 
contains the complete state machine of an actor. I.e. no target language inheritance is used to implement 
the state machine inheritance.
Furthermore the \texttt{ExpandedModelComponent} gives access to
\begin{itemize}
\item \texttt{getIncomingTransitions(StateGraphNode)} -- the set of incoming transition of a 
\texttt{StateGraphNode} (\texttt{State}, \texttt{ChoicePoint} or \texttt{TransitionPoint})
\item \texttt{getOutgoingTransitions(StateGraphNode)} -- the set of outgoing transition of a 
\texttt{StateGraphNode}
\item \texttt{getActiveTriggers(State)} -- the triggers that are active in this \texttt{State} in the 
order they are evaluated
\end{itemize}

\subsubsection*{The Expanded Actor Class}

The \texttt{ExpandedActorClass} is derived from the \texttt{ExpandedModelComponent} and adds only minor new features.
\begin{itemize}
\item \texttt{getActorClass()} -- for convenience to avoid casts of the \texttt{ModelComponent} to an \texttt{ActorClass} 
\item \texttt{getVarDeclData(Transition)} -- for convenience to avoid casts to \texttt{VarDecl}
\end{itemize}

\subsubsection*{Transition Chains}

By transition chains we denote a connected subset of the (hierarchical) state machine that starts with a 
transition starting at a state and continues over transitional state graph nodes (choice points and 
transition points) and continuation transitions until a state is reached. In general a transition chain 
starts at one state and ends in several states (the chain may branch in choice points).
A \texttt{TransitionChain} of a transition is retrieved by a call of \texttt{getChain(Transition)} of the 
\texttt{ExpandedActorClass}.
The \texttt{TransitionChain} accepts an \texttt{ITransitionChainVisitor} which is called along the chain 
to generate the action codes of involved transitions and the conditional statements arising from the 
involved choice points. 

\subsection{Generator Overview}

There is one plug-in that consists of base classes and some generic generator parts which are re-used by 
all language specific generators
 
\subsubsection*{Base Classes and Interfaces}

We just want to mention the most important classes and interfaces.
Some of them can be found in the \texttt{org.eclipse.etrice.generator.fsm} and th rest
in \texttt{org.eclipse.etrice.generator}.

\begin{itemize}
\item \begin{flushleft}\texttt{ITranslationProvider} --- this interface is used by the 
\texttt{DetailCodeTranslator} for the language dependent translation of e.g. port.message() notation in 
detail code\end{flushleft}
\item \texttt{AbstractGenerator} --- concrete language generators should derive from this base class
\item \begin{flushleft}\texttt{DefaultFSMTranslationProvider} and \texttt{DefaultTranslationProvider} --- a stub implementation of 
\texttt{IFSMTranslationProvider} and \texttt{ITranslationProvider} from which clients may derive\end{flushleft}
\item \texttt{Indexed} --- provides an indexed iterable of a given iterable
\item \texttt{GeneratorBaseModule} --- a Google Guice module that binds a couple of basic services. 
Concrete language generators should use a module that derives from this
\end{itemize}

\subsubsection*{Generic Generator Parts}

The generic generator parts provide code generation blocks on a medium granularity. The language dependent 
top level generators embed those blocks in a larger context (file, class, ...). Language dependent low 
level constructs are provided by means of an \texttt{ILanguageExtension}. This extension and other parts 
of the generator be configured using Google Guice dependency injection.

\paragraph*{GenericActorClassGenerator}

The \texttt{GenericActorClassGenerator} generates constants for the interface items of a actor. Those 
constants are used by the generated state machine.

\paragraph*{GenericProtocolClassGenerator}

The \texttt{GenericProtocolClassGenerator} generates message ID constants for a protocol.

\paragraph*{GenericStateMachineGenerator}

\begin{flushleft}The \texttt{GenericStateMachineGenerator} generates the complete state machine 
implementation. The skeleton of the generated code is\end{flushleft}

\begin{itemize}
\item definition state ID constants
\item definition of transition chain constants
\item definition of trigger constants
\item entry, exit and action code methods
\item the \texttt{exitTo} method 
\item the \texttt{executeTransitionChain} method
\item the \texttt{enterHistory} method
\item the \texttt{executeInitTransition} method
\item the \texttt{receiveEvent} method
\end{itemize}

The state machine works as follows. The main entry method is the \\ \texttt{receiveEvent} method. This is 
the case for both, data driven (polled) and event driven state machines. Then a number of nested 
switch/case statements evaluates trigger conditions and derives the transition chain that is executed. If 
a trigger fires then the \texttt{exitTo} method is called to execute all exit codes involved. Then the 
transition chain action codes are executed and the choice point conditions are evaluated in the 
\texttt{executeTransitionChain} method. Finally the history of the state where the chain ends is entered 
and all entry codes are executed by \texttt{enterHistory}.

\subsubsection*{The Java Generator}

The Java generator employs the generic parts of the generator. The \texttt{JavaTranslationProvider} is 
very simple and only handles the case of sending a message from a distinct replicated port: 
\texttt{replPort[2].message()}. Other cases are handled by the base class by returning the original text.

The \texttt{DataClassGen} uses Java inheritance for the generated data classes. Otherwise it is pretty 
much straight forward.

The \texttt{ProtocolClassGen} generates a class for the protocol with nested static classes for regular 
and conjugated ports and similar for replicated ports.

The \texttt{ActorClassGen} uses Java inheritance for the generated actor classes. So ports, SAPs and 
attributes and detail code methods are inherited. Not inherited is the state machine implementation.

\subsubsection*{The ANSI-C Generator}

The C generator translates data, protocol and actor classes into structs together with a set of methods 
that operate on them and receive a pointer to those data (called \texttt{self} in analogy to the implicit 
C++ \texttt{this} pointer).
No dynamic memory allocation is employed. All actor instances are statically initialized.
One of the design goals for the generated C code was an optimized footprint in terms of memory and 
performance to be able to utilize modeling with ROOM also for tiny low end micro controllers.

\subsubsection*{The Documentation Generator}

The documentation generator creates documentation in LaTex format which can be converted into PDF and many 
other formats.

\end{document}