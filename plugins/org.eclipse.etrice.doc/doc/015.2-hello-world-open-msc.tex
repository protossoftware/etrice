\subsection{Open the Message Sequence Chart}

After termination, we can inspect the behavior of our application. It is recorded in the form of a MSC (Message Sequence Chart) and can now be used for debugging or documentation purpose. Open \emph{msc.seq} or \emph{subSystemRef\_Async.seq} in the folder \emph{log} using the tool Trace2UML (if the file is not present, try to refresh (F5) the folder \emph{log}).

\begin{quote}
	If Trace2UML (Open Source tool) is not already installed, it can be obtained here: \href{http://trace2uml.tigris.org/servlets/ProjectDocumentList?folderID=6208}{Windows download site} or \href{http://apt.astade.de/}{Linux package of the Astade UML tool containing Trace2UML}
\end{quote}


Yet the MSC is nearly empty having not recorded any interaction between actors. It shows that  \emph{topActor} (full instance path \emph{/LogSys/subSystemRef/topActor}) has taken \emph{helloState}. In the next PingPong tutorial we are going to create a more sophisticated and vivid \eTrice application introducing actor building blocks and message protocols.

