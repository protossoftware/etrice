\section{Working with the tutorials}

The \eTrice{} tutorials will help you to learn and understand the \eTrice{} tool and its concepts. \eTrice{} supports 
several target languages.

The Hello World tutorial is target language specific. The other tutorials work for all target languages. Target language specific aspects are explained for all languages. 
Currently eTrice supports Java and C. C++ generator and runtime are currently prototypes with no tutorials. You should decide for which target language you want to work through the tutorials. 

\begin{itemize}
	\item Hello World - Getting Started C
	\item Hello World - Getting Started Java
	\item Ping Pong 
	\item Traffic Light (Example)
\end{itemize}

The tutorials are also available in their finished version and can be added to the workspace via the Eclipse New Wizard (\emph{File -> New -> Other: eTrice C/Java Tutorials}).

The \emph{Traffic Light Example} in not yet available but will be provided with the next \eTrice{} milestone.

\eTrice{} generates code out of ROOM models. The generated code relies on the services of a runtime framework (Runtime):
\begin{itemize}
\item execution
\item communication (e.g. messaging)
\item logging
\item operating system abstraction (osal)
\end{itemize}

Additional functionality is provided as model library (Modellib): 
\begin{itemize}
\item socket server and client
\item timing service
\item standard types
\end{itemize}
 